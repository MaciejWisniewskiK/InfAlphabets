\documentclass[12 pt]{article}
\usepackage{amsfonts, amssymb}
  

\oddsidemargin=-0.5cm
\setlength{\textwidth}{6.5in}
\addtolength{\voffset}{-20pt}
\addtolength{\headsep}{25pt}


\pagestyle{myheadings}
\markright{Maciej Wiśniewski \hfill \today \hfill} 


\begin{document}

\section{Problem 1.}
\section{Problem 2.}
Since $X$ is pof, it is orbit-finite. 
Since given relation is equivariant, each orbit of $X$ is either entirely contained within one
equivalence class or each element of the orbit is in a different equivalence class. \\

\noindent
If the first case holds for every orbit, then every equivalence class is a union of orbits, therefore:
\begin{itemize}
    \item There are finitely many equivalence classes.
    \item Each equivalence class is equivariant.
\end{itemize}

\noindent
If the second case holds for at least one orbit, than:
\begin{itemize}
    \item There are infinitely many equivalence classes (because an orbit has one or infinitely many elements and if it had only one, the first case would hold).
    \item Not every equivalence class is equivariant, because elements of the same orbit are by definition equivariantly indistinguishable.
\end{itemize}

$T.H.M.W.$

\section{Problem 3.}

Note that such function $f$ exists if and only if every orbit (orbit generating element) of $X$ is in a relation with an oribt of $Y$ that does not introduce
new atoms (ones that were not used in the $X$'s orbit generating element). \\

\noindent
\textbf{Proof:} If for every relation of given orbit of $X$ ($O$) it would introduce new atoms, than $f$ would have to create new atoms,
and since it has to be equivariant we could only take all atoms or none of them, meaning $f(O)$ doesn't exist or has infinitely many possible values,
therefore $f$ is not a function.\\

\noindent
With that we iterate over all orbits of $X \times Y$ ignoring the ones that are not in $R$ ($R$ is an equivariant subset of $X \times Y$) and
if the given orbit (let's call the $X$ part $O_x$ and the $Y$ part $O_y$) satisfies the condition that $O_y$ does not introduce new atoms, we
mark $O_x$ as an orbit for which $f$ exists. \\

\noindent
Lastly, we check for every orbit of $X$ if $f$ exists for it. Described algorithm will terminate because we operate on pof sets, which are orbit-finite.

$T.H.M.W.$


\end{document}